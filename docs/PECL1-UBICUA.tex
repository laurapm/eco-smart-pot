\documentclass[runningheads]{llncs}
% Hipervínculos
\usepackage{hyperref}
% Tablas
\usepackage[table,xcdraw]{xcolor}
% Idioma y tildes
\usepackage[spanish]{babel}
\selectlanguage{spanish}
\usepackage[utf8]{inputenc}

\usepackage[utf8]{inputenc}

\title{PECL1. COMPUTACIÓN UBICUA} %inserten aqui el nombre de nuestro proyecto
\author{Pablo Acereda García \and David Emanuel Craciunescu \and Pablo Martínez Gracia \and Laura Pérez Medeiro }
\date{October 2019}


\begin{document}

\maketitle

\section{Introducción}

El proyecto consistirá en un sistema inteligente para controlar los factores abioticos que afectan al bienestar de las plantas, con el objetivo de mejorar la salud de las mismas.

Para ello se monitorizarán la humedad y temperatura tanto exterior como de la tierra así como la cantidad de luz que reciben. Con estos valores y una serie de parámetros se generarán las corresponsientes alarmas que indiquen situaciones de riesgo para la salud de la planta.

\section{Contexto}
    \subsection{Situación actual del problema a abordar}
    Actualmente existen bastantes sistemas que se encargan de controlar el riego de las plantas y los cuales ofrecen la posibilidad de ser controlados mediante dispositivos móviles. Algunos de ellos son Blossom, PlantLink o Edyn. Incluso encontramos maceteros inteligentes, como el que nos ofrece Xiaomi, capaz de regar las plantas automáticamente.También encontramos otros proyectos como GRO, capaz de sugerirnos especies de plantas que podríamos cultivar en función del terreno que disponemos.
    \newline
    Sin embargo, ninguno nos ofrece integración con ningún asistente virtual como pueda ser Alexa, ese será nuestro principal objetivo.
    
    \subsection{Situación prevista al final del proyecto}
    Al finalizar la asignatura se pretende tener un dispositivo capaz de mantener con vida un cultivo de manera automática y el cual pueda ser controlado mediante comandos de voz gracias a Alexa.
    Para ello, nuestro dispositivo contará con la predicción del tiempo, aviso de posibles plagas, estado del terreno y condiciones óptimas necesarias para el tipo de planta que se posea.
    
    \subsection{Beneficiarios del proyecto}
    Con este proyecto se pretende ayudar a los aficionados de la jardinería que no disponen de una gran cantidad de tiempo para el óptimo cuidado de las plantas, así como a agricultores que necesiten una ayuda extra para minimizar el impacto de situaciones climáticas extremas. Un ejemplo de situación  climáticas extrema que este proyecto pretende minimizar, son las grandes heladas en momentos inesperados. 
    
\section{Misión y alcance del proyecto}
En los apartados anteriores se ha empezado a comentar estos aspectos, pero ahora se detallará con un mayor nivel de detalle:


\end{document}
