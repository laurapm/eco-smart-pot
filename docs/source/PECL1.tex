%%%%%%%%%%%%%%%%%%%%%%%%%%%%%%%%%%%%%%%%%%%%%%%%%%%%%%%%%%%%%%%%%%%%%%%%%%%%%%%%
% Ubiquitous Computation - Lab Practice 1                                      %
%   - Pablo Acereda García                                                     %
%   - David Emanuel Craciunescu                                                %
%   - Pablo Martínez Gracia                                                    %
%   - Laura Pérez Medeiro                                                      %
%%%%%%%%%%%%%%%%%%%%%%%%%%%%%%%%%%%%%%%%%%%%%%%%%%%%%%%%%%%%%%%%%%%%%%%%%%%%%%%%

% This template has been tested with LLNCS DOCUMENT CLASS -- version 2.20 (10-Mar-2018)

% !TeX spellcheck   = en-US
% !TeX encoding     = utf8
% !TeX program      = pdflatex
% !BIB program      = bibtex
% -*- coding:utf-8 mod:LaTeX -*-

% "a4paper" enables:
%  - easy print out on DIN A4 paper size
%

% English documents: pass english as class option
\documentclass[english,runningheads,a4paper]{llncs}[2018/03/10]
\usepackage[ngerman,main=english]{babel}
\addto\extrasenglish{\languageshorthands{ngerman}\useshorthands{"}}

\usepackage{regexpatch}
\makeatletter
\edef\switcht@albion{%
  \relax\unexpanded\expandafter{\switcht@albion}%
}
\xpatchcmd*{\switcht@albion}{ \def}{\def}{}{}
\xpatchcmd{\switcht@albion}{\relax}{}{}{}
\edef\switcht@deutsch{%
  \relax\unexpanded\expandafter{\switcht@deutsch}%
}
\xpatchcmd*{\switcht@deutsch}{ \def}{\def}{}{}
\xpatchcmd{\switcht@deutsch}{\relax}{}{}{}
\edef\switcht@francais{%
  \relax\unexpanded\expandafter{\switcht@francais}%
}
\xpatchcmd*{\switcht@francais}{ \def}{\def}{}{}
\xpatchcmd{\switcht@francais}{\relax}{}{}{}
\makeatother

\usepackage{ifluatex}
\ifluatex
  \usepackage{fontspec}
  \usepackage[english]{selnolig}
\fi

\iftrue % use default-font
  \ifluatex
    \setmainfont{Latin Modern Roman}
    \setsansfont{Latin Modern Sans}
    \setmonofont{Latin Modern Mono} % "variable=false"
  \else
    \usepackage[%
      rm={oldstyle=false,proportional=true},%
      sf={oldstyle=false,proportional=true},%
      tt={oldstyle=false,proportional=true,variable=false},%
      qt=false%
    ]{cfr-lm}
  \fi
\else
  \ifluatex
    \setmainfont{TeX Gyre Termes}
    \setsansfont[Scale=.9]{TeX Gyre Heros}
    \setmonofont{Latin Modern Mono} % "variable=false"
  \else
    \usepackage{newtxtext}
    \usepackage{newtxmath}
    \usepackage[zerostyle=b,scaled=.9]{newtxtt}
  \fi
\fi

\ifluatex
\else
  \usepackage[T1]{fontenc}
  \usepackage[utf8]{inputenc} %support umlauts in the input
\fi

\usepackage{graphicx}
\usepackage{upquote}
\usepackage{booktabs}
\usepackage{paralist}
\usepackage{csquotes}
\usepackage{textcmds}

% Enable margin kerning
\RequirePackage[%
  babel,%
  final,%
  expansion=alltext,%
  protrusion=alltext-nott]{microtype}%
\DisableLigatures{encoding = T1, family = tt* }

\usepackage{url}
\makeatletter
\g@addto@macro{\UrlBreaks}{\UrlOrds}
\makeatother

% Required for package pdfcomment later
\usepackage{xcolor}

% For listings
\usepackage{listings}
\lstset{%
  basicstyle=\ttfamily,%
  columns=fixed,%
  basewidth=.5em,%
  xleftmargin=0.5cm,%
  captionpos=b}%
\renewcommand{\lstlistingname}{List.}
\usepackage{chngcntr}
\AtBeginDocument{\counterwithout{lstlisting}{section}}

% Enable nice comments
\usepackage{pdfcomment}

\newcommand{\commentontext}[2]{\colorbox{yellow!60}{#1}\pdfcomment[color={0.234 0.867 0.211},hoffset=-6pt,voffset=10pt,opacity=0.5]{#2}}
\newcommand{\commentatside}[1]{\pdfcomment[color={0.045 0.278 0.643},icon=Note]{#1}}

\newcommand{\todo}[1]{\commentatside{#1}}
% Compatiblity with package fixmetodonotes
\newcommand{\TODO}[1]{\commentatside{#1}}

% Bibliography
\ifluatex
\else
  \SetExpansion
  [ context = sloppy,
    stretch = 30,
    shrink = 60,
    step = 5 ]
  { encoding = {OT1,T1,TS1} }
  { }
\fi

% Put footnotes below floats
% Source: https://tex.stackexchange.com/a/32993/9075
\usepackage{stfloats}
\fnbelowfloat

\usepackage{hyperref}
\hypersetup{hidelinks,
  colorlinks=true,
  allcolors=black,
  pdfstartview=Fit,
  breaklinks=true}

% Enable correct jumping to figures when referencing
\usepackage[all]{hypcap}

\usepackage[group-four-digits,per-mode=fraction]{siunitx}

\usepackage[capitalise,nameinlink]{cleveref}

% Nice formats for \cref
\usepackage{iflang}
\IfLanguageName{ngerman}{
  \crefname{table}{Tab.}{Tab.}
  \Crefname{table}{Tabelle}{Tabellen}
  \crefname{figure}{\figurename}{\figurename}
  \Crefname{figure}{Abbildungen}{Abbildungen}
  \crefname{equation}{Gleichung}{Gleichungen}
  \Crefname{equation}{Gleichung}{Gleichungen}
  \crefname{listing}{\lstlistingname}{\lstlistingname}
  \Crefname{listing}{Listing}{Listings}
  \crefname{section}{Abschnitt}{Abschnitte}
  \Crefname{section}{Abschnitt}{Abschnitte}
  \crefname{paragraph}{Abschnitt}{Abschnitte}
  \Crefname{paragraph}{Abschnitt}{Abschnitte}
  \crefname{subparagraph}{Abschnitt}{Abschnitte}
  \Crefname{subparagraph}{Abschnitt}{Abschnitte}
}{
  \crefname{section}{Sect.}{Sect.}
  \Crefname{section}{Section}{Sections}
  \crefname{listing}{\lstlistingname}{\lstlistingname}
  \Crefname{listing}{Listing}{Listings}
}

% Solution for hyperlink refs.
\newcommand{\Vlabel}[1]{\label[line]{#1}\hypertarget{#1}{}}
\newcommand{\lref}[1]{\hyperlink{#1}{\FancyVerbLineautorefname~\ref*{#1}}}
    
\usepackage{xspace}
%\newcommand{\eg}{e.\,g.\xspace}
%\newcommand{\ie}{i.\,e.\xspace}
\newcommand{\eg}{e.\,g.,\ }
\newcommand{\ie}{i.\,e.,\ }

% Powerset
\DeclareFontFamily{U}{MnSymbolC}{}
\DeclareSymbolFont{MnSyC}{U}{MnSymbolC}{m}{n}
\DeclareFontShape{U}{MnSymbolC}{m}{n}{
  <-6>    MnSymbolC5
  <6-7>   MnSymbolC6
  <7-8>   MnSymbolC7
  <8-9>   MnSymbolC8
  <9-10>  MnSymbolC9
  <10-12> MnSymbolC10
  <12->   MnSymbolC12%
}{}
\DeclareMathSymbol{\powerset}{\mathord}{MnSyC}{180}

% Name says it all.
\ifluatex
\else
  \input glyphtounicode
  \pdfgentounicode=1
\fi

% Correct bad hypenation.
\hyphenation{op-tical net-works semi-conduc-tor}

% Some useful info to add.

\iffalse
  \usepackage[intended]{llncsconf}
  \conference{name of the conference}
  \llncs{book editors and title}{0042} %% 0042 is the start page
\fi

% For demonstration purposes only
\usepackage[math]{blindtext}
\usepackage{mwe}
\usepackage[backend=biber, style=numeric]{biblatex}
\addbibresource{java.bib}
\usepackage[ampersand]{easylist}

%%%%%%%%%%%%%%%%%%%%%%%%%%%%%%%%%%%%%%%%%%%%%%%%%%%%%%%%%%%%%%%%%%%%%%%%%%%%%%%%

\title{ec\textbf{\o}}
\author{
    Pablo Acereda Gracia \and
    David Emanuel Craciunesuc \and
    Pablo Martínes García \and
    Laura Pérez Medeiro
}

\date{October 2019}

\begin{document}

\maketitle

%%%%%%%%%%%%%%%%%%%%%%%%%%%%%%%%%%%%%%%%%%%%%%%%%%%%%%%%%%%%%%%%%%%%%%%%%%%%%%%%

\section*{Introduction}

This project will consist of an intelligent system with capabilities to control
abiotic factors that affect the wellbeing of plants, with the objective of
improviing their health and quality of life.

Factors such as humidity and temperature, as well as the levels of light these
receive, will be monitorized and analyzed frequently in order to verify and
ensure optimal quality of life for the plants.

Thanks to these measurements, a series of alarms and different warning states
will be implemented in order to better control the status of the plants and
alert the users of their current situation.

Users will be able to easily install the system themselves, and interact and
control it via a virtual assistant such as \textit{Amazon Alexa}.

%%%%%%%%%%%%%%%%%%%%%%%%%%%%%%%%%%%%%%%%%%%%%%%%%%%%%%%%%%%%%%%%%%%%%%%%%%%%%%%%

\section*{Context}

    %%%%%%%%%%%%%%%%%%%%%%%%%%%%%%%%%%%%

    \subsection*{Current situation of the presented problem}

    The market has seen its fair share of intelligent systems created to aid
    with plant irrigation control, and there are those that even come with a
    mobile app interface, such as Blossom, PlantLink or Edyn. Some of the
    newcomers to jump aboard the backyard-sprinkler train are the so-called
    `intelligent flowerpots', like the one \textit{Xiaomi} has recently started
    to sell, capable of watering the plants automatically depending on the
    humidity of the soil.

    There are other projects like \textit{GR0} that are capable of suggesting
    what plant species one should buy by analyzing the quality and type of soil
    one uses.

    Nevertheless, none of these offers use any kind of virtual-assistant
    integration or any well-designed user experience, for that matter.
    \textit{That} will be the main difference our project will have. Not only
    will the user control the system through a virtual assistant, the design and
    user experience is planned to be exceptional and extremely easy and
    intuitive.

    %%%%%%%%%%%%%%%%%%%%%%%%%%%%%%%%%%%%

    \subsection*{End-of-project prediction}

    Once the course finishes, our intention is to have created a device capable
    of auomatically keeping alive crops or plants with interactions through
    \textit{Amazon Alexa}.

    In order to achieve that, our device will be able to give accurate weather
    predictions, alert of possible plagues that might attack the crops and
    monitor the optimal conditions for the plants themselves.

    %%%%%%%%%%%%%%%%%%%%%%%%%%%%%%%%%%%%

    \subsection*{Target audience}

    This project aims to aid the gardening aficionados that do not have a great
    amount of time at their disposal to take care of their plants optimally. It
    also aims to assist farmers that need help with the care of their crops and
    seek to minimize the effect of unexpected and external factors to their
    produce.

%%%%%%%%%%%%%%%%%%%%%%%%%%%%%%%%%%%%%%%%%%%%%%%%%%%%%%%%%%%%%%%%%%%%%%%%%%%%%%%%

\section*{Project Scope}

%%%%%%%%%%%%%%%%%%%%%%%%%%%%%%%%%%%%%%%%%%%%%%%%%%%%%%%%%%%%%%%%%%%%%%%%%%%%%%%%

\section*{Discarded Ideas}

%%%%%%%%%%%%%%%%%%%%%%%%%%%%%%%%%%%%%%%%%%%%%%%%%%%%%%%%%%%%%%%%%%%%%%%%%%%%%%%%

\section*{Technology to Use}

This section contains the different options that were considered for the
project. The different decisions about the different used technology within ght
project took into account the experience of the various members of the group, as
well as the monetary cost of the different hardware elements, their general
availability and the trust the very manufacturer inspired.

%%%%%%%%%%%%%%%%%%%%%%%%%%%%%%%%%%%%%%%%

\subsection*{Controllers}

\begin{easylist}[itemize]

& \textbf{Arduino}

\textit{Arduino} is an open-source electronics platform that is based on
easy-to-use software and hardware. The \textit{Arduino} boards themselves are
controlled by sending a set of instructions to a microcontroller on the board.
To do so, one must use the \textit{Arduino} programming language, which is based
on \textit{Wiring}, and the \textbf{Arduino Software (IDE)}, which is based on
\textit{Processing}.

These boards were our first choice because they are extremely easy to use. The
programming language is extremely easy to pick up if one already knows
\textit{C}, and the \textit{Arduino} community offers a wide array of
free-to-use resources that improve the quality and reduce the effort of any project trying
to use \textit{Arduino}.

Even with all it's benefits, \textit{Arduino} ended up being discarded as a
possible option, given that the absolutely cheapest of boards would still cost
around \$20.

& \textbf{Raspberry Pi}

According to the official \textit{Raspberry Pi} webpage: \textit{``Raspberry Pi
is a low cost, credit-card sized computer that plugs into a monitor or a TV, and
uses standard keyboard and mouse. (\ldots) It's capable of doing what you'd
normally expect a regular computer to do, from browsing the internet and playing
high-definition video, to making spreadsheets, word processing and playing
games''}.

Just like a regular computer, once would be able to program it to do whatever
they'd want it to do. This would have been ideal for the project itself, given
the simplicity of its use and the gigantic array of languages that are
compatible with it. In the end, though, it was discarded because the computer
itself was too large in comparison to the rest of the elements of the project.

& \textbf{ESP32}

The \textit{ESP32} is a series of low-cost, low-power system on a chip
\textbf{SoC} microcontrollers with integrated Wi-Fi and dual-mode Bluetooth.
This was the option the team ended up choosing for the project, given that the
\textit{ESP32} was specifically designed for \textbf{wearable electronics and
IoT applications.}

One can also program on it using the \textit{Arduino IDE}, which was a huge
advantage. Most of the team already knew how to use and control
\textit{Arduino}, so not having to learn a skil specifically for the
microcontroller plus all the benefits of the \textit{Arduino} community made the
group decide on the \textit{ESP32} as an option.

\end{easylist}

%%%%%%%%%%%%%%%%%%%%%%%%%%%%%%%%%%%%%%%%

\subsection*{Sensors}

\begin{easylist}[itemize]

& DHT22
& DSB18B20:
& LDR:
& YL-69:

\end{easylist}

%%%%%%%%%%%%%%%%%%%%%%%%%%%%%%%%%%%%%%%%

\subsection*{Programming Languages}

As previously mentioned, the chosen microcontroller is the \textit{ESP32}.
Therefore, the options when it comes to programming languages have been:

\begin{easylist}[itemize]
    
& MicroPython

    \textit{MicroPython} an open source \textit{Python} programming language
    interpreter that is capable of running on small embedded development boards.
    With the adaptability build into \textit{MicroPython}, one can write clean
    and simple \textit{Python} code to control hardware directly instead of
    having to use complex low-level languages.

    Even if a reduced version, \textit{MicroPython} still supports most of
    \textit{Python}'s syntax and implements most of its inner mechanism. Given
    all these features and advantages, it was a quick initial choice for the
    project.

    These are some of the features that set it apart from other embedded
    systems:

    && \textbf{Interactive REPL}
    This feature allows execution of code without the need of any compilation or
    uploading time, which is perfect for systems with a high level of
    experimentation.

    && \textbf{Extensive software library}
    \textit{MicroPython} already comes with libraries built in to support common
    tasks, like JSON data parsing, regular expression handling or even network
    socket programming.

    && \textbf{Extensibility}
    Advanced users ma be able to mix \textit{MicroPython} with extensible
    low-level C/C++ functions in order to further optimize their code and make
    the execution faster when it really matters.

    \textit{MicroPython} has some very useful features. Unfortunately, it also
    comes with its downsides:

    && Slower code and higher memory needs when compared to C/C++.
    && Complicated microcontroller initialization process.
    && Limited functionality for some key libraries.

& FreeRTOS

\textit{FreeRTOS} is a real-time operating system made specifically to run on
embedded systems. It is especially good because it natively provides core real
time scheduling, inter-task communication, timing and syncrhonisation
primitives. This transalates into a more accurately controlled kernel and a
system that is able to execute tasks exactly when they have to be executed, a
\textit{deterministic} system.

In \textit{FreeRTOS}, applications can be assigned in a static manner while
objects themselves can be assigned dynamically with different memory-assignment
memory schemes. This system was quickly discarded. Even with the great deal of
documentation it counts, the complexity of \textit{FreeRTOS} and the learning
curve it would entail made it practiacally impossible in the provided timeframe.

& Mongoose OS

\textit{Mongoose OS} is an Internet of Things Firmware Development Framework
under Apache License Version 2.0.

\textit{Mongoose OS} is a firmware development framework specifically designed
for development of IoT products. It is highly compatible with a wide array of
microcontrollers, just like the \textit{ESP32}, and it's objective is to fill
`the noticeable for embedded software developers' between firmware created for
prototyping and bare-metal microcontrollers.

It comes with an integrated web server and supports interaction with both
private and public clouds like AWS IoT or Mosquitto.

& Arduino IDE:

\end{easylist}

%%%%%%%%%%%%%%%%%%%%%%%%%%%%%%%%%%%%%%%%%%%%%%%%%%%%%%%%%%%%%%%%%%%%%%%%%%%%%%%%

\section*{Software Development Methodologies}

%%%%%%%%%%%%%%%%%%%%%%%%%%%%%%%%%%%%%%%%%%%%%%%%%%%%%%%%%%%%%%%%%%%%%%%%%%%%%%%%

\section*{Application Architecture}

%%%%%%%%%%%%%%%%%%%%%%%%%%%%%%%%%%%%%%%%%%%%%%%%%%%%%%%%%%%%%%%%%%%%%%%%%%%%%%%%

\section*{Business Model}

%%%%%%%%%%%%%%%%%%%%%%%%%%%%%%%%%%%%%%%%%%%%%%%%%%%%%%%%%%%%%%%%%%%%%%%%%%%%%%%%

\section*{Development Plan}

%%%%%%%%%%%%%%%%%%%%%%%%%%%%%%%%%%%%%%%%%%%%%%%%%%%%%%%%%%%%%%%%%%%%%%%%%%%%%%%%

\section*{Risk Assessment}

%%%%%%%%%%%%%%%%%%%%%%%%%%%%%%%%%%%%%%%%%%%%%%%%%%%%%%%%%%%%%%%%%%%%%%%%%%%%%%%%

\section*{Contingency Plan}

%%%%%%%%%%%%%%%%%%%%%%%%%%%%%%%%%%%%%%%%%%%%%%%%%%%%%%%%%%%%%%%%%%%%%%%%%%%%%%%%

\section*{Overview of the Project}

%%%%%%%%%%%%%%%%%%%%%%%%%%%%%%%%%%%%%%%%%%%%%%%%%%%%%%%%%%%%%%%%%%%%%%%%%%%%%%%%

\printbibliography

\end{document}
